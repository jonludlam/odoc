\section{Module \ocamlinlinecode{Functor2}}\label{container-page-test-module-Functor2}%
\label{container-page-test-module-Functor2-module-type-S}\ocamlcodefragment{\ocamltag{keyword}{module} \ocamltag{keyword}{type} \hyperref[container-page-test-module-Functor2-module-type-S]{\ocamlinlinecode{S}}}\ocamlcodefragment{ = \ocamltag{keyword}{sig}}\begin{ocamlindent}\label{container-page-test-module-Functor2-module-type-S-type-t}\ocamlcodefragment{\ocamltag{keyword}{type} t}\\
\end{ocamlindent}%
\ocamlcodefragment{\ocamltag{keyword}{end}}\\
\label{container-page-test-module-Functor2-module-X}\ocamlcodefragment{\ocamltag{keyword}{module} \hyperref[container-page-test-module-Functor2-module-X]{\ocamlinlinecode{X}}}\ocamlcodefragment{ (\hyperref[container-page-test-module-Functor2-module-X-argument-1-Y]{\ocamlinlinecode{Y}} : \hyperref[container-page-test-module-Functor2-module-type-S]{\ocamlinlinecode{S}}) (\hyperref[container-page-test-module-Functor2-module-X-argument-2-Z]{\ocamlinlinecode{Z}} : \hyperref[container-page-test-module-Functor2-module-type-S]{\ocamlinlinecode{S}}) : \ocamltag{keyword}{sig} .\allowbreak{}.\allowbreak{}.\allowbreak{} \ocamltag{keyword}{end}}\\
\label{container-page-test-module-Functor2-module-type-XF}\ocamlcodefragment{\ocamltag{keyword}{module} \ocamltag{keyword}{type} \hyperref[container-page-test-module-Functor2-module-type-XF]{\ocamlinlinecode{XF}}}\ocamlcodefragment{ = \ocamltag{keyword}{sig}}\begin{ocamlindent}\subsubsection{Parameters\label{parameters}}%
\label{container-page-test-module-Functor2-module-type-XF-argument-1-Y}\ocamlcodefragment{\ocamltag{keyword}{module} \hyperref[container-page-test-module-Functor2-module-type-XF-argument-1-Y]{\ocamlinlinecode{Y}}}\ocamlcodefragment{ : \ocamltag{keyword}{sig}}\begin{ocamlindent}\label{container-page-test-module-Functor2-module-type-XF-argument-1-Y-type-t}\ocamlcodefragment{\ocamltag{keyword}{type} t}\\
\end{ocamlindent}%
\ocamlcodefragment{\ocamltag{keyword}{end}}\\
\label{container-page-test-module-Functor2-module-type-XF-argument-2-Z}\ocamlcodefragment{\ocamltag{keyword}{module} \hyperref[container-page-test-module-Functor2-module-type-XF-argument-2-Z]{\ocamlinlinecode{Z}}}\ocamlcodefragment{ : \ocamltag{keyword}{sig}}\begin{ocamlindent}\label{container-page-test-module-Functor2-module-type-XF-argument-2-Z-type-t}\ocamlcodefragment{\ocamltag{keyword}{type} t}\\
\end{ocamlindent}%
\ocamlcodefragment{\ocamltag{keyword}{end}}\\
\subsubsection{Signature\label{signature}}%
\label{container-page-test-module-Functor2-module-type-XF-type-y+u+t}\ocamlcodefragment{\ocamltag{keyword}{type} y\_\allowbreak{}t = \hyperref[container-page-test-module-Functor2-module-type-XF-argument-1-Y-type-t]{\ocamlinlinecode{Y.\allowbreak{}t}}}\\
\label{container-page-test-module-Functor2-module-type-XF-type-z+u+t}\ocamlcodefragment{\ocamltag{keyword}{type} z\_\allowbreak{}t = \hyperref[container-page-test-module-Functor2-module-type-XF-argument-2-Z-type-t]{\ocamlinlinecode{Z.\allowbreak{}t}}}\\
\label{container-page-test-module-Functor2-module-type-XF-type-x+u+t}\ocamlcodefragment{\ocamltag{keyword}{type} x\_\allowbreak{}t = \hyperref[container-page-test-module-Functor2-module-type-XF-type-y+u+t]{\ocamlinlinecode{y\_\allowbreak{}t}}}\\
\end{ocamlindent}%
\ocamlcodefragment{\ocamltag{keyword}{end}}\\

\section{Module \ocamlinlinecode{Functor2.\allowbreak{}X}}\label{container-page-test-module-Functor2-module-X}%
\subsection{Parameters\label{parameters}}%
\label{container-page-test-module-Functor2-module-X-argument-1-Y}\ocamlcodefragment{\ocamltag{keyword}{module} \hyperref[container-page-test-module-Functor2-module-X-argument-1-Y]{\ocamlinlinecode{Y}}}\ocamlcodefragment{ : \ocamltag{keyword}{sig}}\begin{ocamlindent}\label{container-page-test-module-Functor2-module-X-argument-1-Y-type-t}\ocamlcodefragment{\ocamltag{keyword}{type} t}\\
\end{ocamlindent}%
\ocamlcodefragment{\ocamltag{keyword}{end}}\\
\label{container-page-test-module-Functor2-module-X-argument-2-Z}\ocamlcodefragment{\ocamltag{keyword}{module} \hyperref[container-page-test-module-Functor2-module-X-argument-2-Z]{\ocamlinlinecode{Z}}}\ocamlcodefragment{ : \ocamltag{keyword}{sig}}\begin{ocamlindent}\label{container-page-test-module-Functor2-module-X-argument-2-Z-type-t}\ocamlcodefragment{\ocamltag{keyword}{type} t}\\
\end{ocamlindent}%
\ocamlcodefragment{\ocamltag{keyword}{end}}\\
\subsection{Signature\label{signature}}%
\label{container-page-test-module-Functor2-module-X-type-y+u+t}\ocamlcodefragment{\ocamltag{keyword}{type} y\_\allowbreak{}t = \hyperref[container-page-test-module-Functor2-module-X-argument-1-Y-type-t]{\ocamlinlinecode{Y.\allowbreak{}t}}}\\
\label{container-page-test-module-Functor2-module-X-type-z+u+t}\ocamlcodefragment{\ocamltag{keyword}{type} z\_\allowbreak{}t = \hyperref[container-page-test-module-Functor2-module-X-argument-2-Z-type-t]{\ocamlinlinecode{Z.\allowbreak{}t}}}\\
\label{container-page-test-module-Functor2-module-X-type-x+u+t}\ocamlcodefragment{\ocamltag{keyword}{type} x\_\allowbreak{}t = \hyperref[container-page-test-module-Functor2-module-X-type-y+u+t]{\ocamlinlinecode{y\_\allowbreak{}t}}}\\



